\documentclass[xcolor=tex,dvipsnames]{beamer}  % for hardcopy add 'trans'

\mode<presentation>
{
  \usetheme{Singapore}
  % or ...
  \setbeamercovered{transparent}
  % or whatever (possibly just delete it)
}


\usepackage{fontspec} 
\usepackage[mathscr]{euscript} 
%\usepackage[xcharter]{newtxmath}
%\setmainfont{XCharter}
\setmonofont{DejaVu Sans Mono}[Scale=MatchLowercase] % provides unicode characters 


\usefonttheme{professionalfonts}
%\usepackage[english]{babel}
% or whatever
%\usepackage[latin1]{inputenc}
% or whatever
%\usepackage{times}
\usepackage[T1]{fontenc}
% Or whatever. Note that the encoding and the font should match. If T1
% does not look nice, try deleting the line with the fontenc.

%%%%%%%%%%%%%%%%%%%%%% start my preamble %%%%%%%%%%%%%%%%%%%%%%
\renewcommand{\insertnavigation}[1]{}

\addtobeamertemplate{navigation symbols}{}{%
    \usebeamerfont{footline}%
    \usebeamercolor[fg]{footline}%
    \hspace{1em}%
    \insertframenumber/\inserttotalframenumber
}

\setbeamercolor{footline}{fg=blue}
\setbeamerfont{footline}{series=\bfseries}

%\setbeamertemplate{mini frames}{}

\usepackage{adjustbox}

\usepackage{graphicx}
\usepackage{amsmath, amssymb, amsthm}

\usepackage{fancyvrb}

\usepackage{hyperref}

% fonts, caligraphic
\usepackage{mathpazo}
%\usepackage{mathrsfs}
\usepackage{bbm}

% tikz
\usepackage{tikz}
\usetikzlibrary{positioning}
\usetikzlibrary{arrows}
\usetikzlibrary{calc}
\usetikzlibrary{intersections}
\usetikzlibrary{matrix}
\usetikzlibrary{decorations}
\usepackage{pgf}
\usepackage{pgfplots}
\usetikzlibrary{shapes, fit}

% from fazeleh
\usetikzlibrary{arrows.meta}

%\usetikzlibrary{arrows.meta}
\usetikzlibrary{decorations.pathreplacing}  %for brac



\usepackage{graphviz}

%\usepackage[usenames, dvipsnames]{color}


% nice inequalities
\renewcommand{\leq}{\leqslant}
\renewcommand{\geq}{\geqslant}


\setlength{\parskip}{1.5ex plus0.5ex minus0.5ex}
\setlength{\jot}{12pt} 

\usepackage[ruled, lined]{algorithm2e}


\definecolor{pale}{RGB}{235, 235, 235}
\definecolor{pale2}{RGB}{175,238,238}
\definecolor{turquois4}{RGB}{0,134,139}
\definecolor{DarkOrange1}{RGB}{255,127,0}

\newcommand{\emp}[1]{\textcolor{DarkOrange1}{\bf #1}}
\newcommand{\newtopic}[1]{\textcolor{Green}{\Large \bf #1}}
\newcommand{\navy}[1]{\textcolor{Blue}{\bf #1}}
\newcommand{\navymth}[1]{\textcolor{Blue}{#1}}
\newcommand{\red}[1]{\textcolor{red}{#1}}
\newcommand{\brown}[1]{\textcolor{Brown}{\sf #1}}
\newcommand{\green}[1]{\textcolor{ForestGreen}{\sf #1}}

% Minted
\definecolor{bg}{rgb}{0.95,0.95,0.95}
\usepackage{minted}
\usemintedstyle{friendly}
\newminted{python}{mathescape,frame=lines,framesep=4mm,bgcolor=bg}
\newminted{ipython}{mathescape,frame=lines,framesep=4mm,bgcolor=bg}
\newminted{julia}{mathescape,frame=lines,framesep=4mm,bgcolor=bg}
\newminted{c}{mathescape,frame=lines,framesep=4mm,bgcolor=bg}
\renewcommand{\theFancyVerbLine}{\sffamily
    \textcolor[rgb]{0.5,0.5,1.0}{\scriptsize {\arabic{FancyVerbLine}}}}


\newcommand{\Fact}{\textcolor{Brown}{\bf Fact. }}
\newcommand{\Facts}{\textcolor{Brown}{\bf Facts }}
\newcommand{\keya}{\textcolor{turquois4}{\bf Key Idea. }}
\newcommand{\Factnodot}{\textcolor{Brown}{\bf Fact }}
\newcommand{\Eg}{\textcolor{ForestGreen}{Example. }}
\newcommand{\Egs}{\textcolor{ForestGreen}{Examples. }}
\newcommand{\Ex}{{\bf Ex. }}



\newcommand{\CC}{\mathbbm C}
\newcommand{\EE}{\mathbbm E}
\newcommand{\FF}{\mathbbm F}
\newcommand{\RR}{\mathbbm R}
\newcommand{\KK}{\mathbbm K}
\newcommand{\MM}{\mathbbm M}
\newcommand{\NN}{\mathbbm N}
\newcommand{\PP}{\mathbbm P}
\newcommand{\TT}{\mathbbm T}
\newcommand{\QQ}{\mathbbm Q}
\newcommand{\WW}{\mathbbm W}
\newcommand{\VV}{\mathbbm V}
\newcommand{\ZZ}{\mathbbm Z}

\newcommand{\Asf}{\mathsf A}
\newcommand{\Esf}{\mathsf E}
\newcommand{\Fsf}{\mathsf F}
\newcommand{\Gsf}{\mathsf G}
\newcommand{\Msf}{\mathsf M}
\newcommand{\Lsf}{\mathsf L}
\newcommand{\Nsf}{\mathsf N}
\newcommand{\Psf}{\mathsf P}
\newcommand{\Qsf}{\mathsf Q}
\newcommand{\Ssf}{\mathsf S}
\newcommand{\Tsf}{\mathsf T}
\newcommand{\Xsf}{\mathsf X}
\newcommand{\Ysf}{\mathsf Y}
\newcommand{\Vsf}{\mathsf V}
\newcommand{\Wsf}{\mathsf W}
\newcommand{\Zsf}{\mathsf Z}

\newcommand{\aA}{\mathscr A}
\newcommand{\bB}{\mathscr B}
\newcommand{\cC}{\mathscr C}
\newcommand{\dD}{\mathscr D}
\newcommand{\eE}{\mathscr E}
\newcommand{\gG}{\mathscr G}
\newcommand{\hH}{\mathscr H}
\newcommand{\iI}{\mathscr I}
\newcommand{\fF}{\mathscr F}
\newcommand{\lL}{\mathscr L}
\newcommand{\pP}{\mathscr P}
\newcommand{\rR}{\mathscr R}
\newcommand{\sS}{\mathscr S}
\newcommand{\vV}{\mathscr V}
\newcommand{\wW}{\mathscr W}
\newcommand{\mM}{\mathscr M}
\newcommand{\oO}{\mathscr O}
\newcommand{\zZ}{\mathscr Z}

\newcommand{\bP}{\mathbf P} 
\newcommand{\bR}{\mathbf R}
\newcommand{\bQ}{\mathbf Q}


%%%%%% special symbols %%%%%%%%%%

\newcommand{\volone}{Volume~I}
\newcommand{\voltwo}{Volume~II}

% transpose, not currently using
\newcommand\T{{\mathpalette\raiseT\intercal}}
\newcommand\raiseT[2]{\raisebox{0.25ex}{$#1#2$}}

% nice inequalities
\renewcommand{\leq}{\leqslant}
\renewcommand{\geq}{\geqslant}

% nice greek letters
\renewcommand{\phi}{\varphi}
\renewcommand{\epsilon}{\varepsilon}

% inner product
\providecommand{\inner}[1]{\left\langle{#1}\right\rangle}

% set of matrices
\newcommand{\matset}[2]{ \MM^{ #1 \times #2 } }

% stochastic dominance
\newcommand{\lefsd}{\preceq_{\textrm{F}}}
\newcommand{\lessd}{\preceq_{\textrm{S}}}

% argmax and min
\newcommand{\argmax}{\operatornamewithlimits{argmax}}
\newcommand{\argmin}{\operatornamewithlimits{argmin}}

% sets and logic
\newcommand{\st}{\ensuremath{\ \mathrm{s.t.}\ }}
\newcommand{\setntn}[2]{ \{ #1 : #2 \} }
\newcommand{\natset}[1]{[ #1 ]}

% some useful symbols
\newcommand{\given}{\, | \,}
\newcommand{\cf}[1]{ \lstinline|#1| }
\newcommand{\fore}{\therefore \quad}
\newcommand{\1}{\mathbbm 1}
\newcommand{\me}{\mathrm{e}}               % Euler's e
\newcommand*\diff{\mathop{}\!\mathrm{d}}   % d for integrals

% shortcuts
\newcommand{\la}{\langle}
\newcommand{\ra}{\rangle}

% relations
\newcommand{\eqdist}{\stackrel{d} {=} }
\newcommand{\iidsim}{\stackrel{\textrm{ {\sc iid }}} {\sim} }

% convergence
\newcommand{\tod}{\stackrel { d } {\to} }
\newcommand{\tow}{\stackrel { w } {\to} }
\newcommand{\toprob}{\stackrel { p } {\to} }
\newcommand{\toms}{\stackrel { ms } {\to} }


%%%%%%%%%%% operators %%%%%%%%%%%%

\DeclareMathOperator{\Exp}{Exp}  % exponential draw
\DeclareMathOperator{\Lip}{Lip}
\DeclareMathOperator{\cl}{cl}
\DeclareMathOperator{\graph}{graph}
\DeclareMathOperator{\interior}{int}
\DeclareMathOperator{\Prob}{Prob}
\DeclareMathOperator{\determinant}{det}
\DeclareMathOperator{\trace}{trace}
\DeclareMathOperator{\sgn}{sgn}
\DeclareMathOperator{\Span}{span}
\DeclareMathOperator{\diag}{diag}
\DeclareMathOperator{\proj}{proj}
\DeclareMathOperator{\rank}{rank}
\DeclareMathOperator{\kernel}{null}
\DeclareMathOperator{\cov}{Cov}
\DeclareMathOperator{\corr}{Corr}
\DeclareMathOperator{\var}{Var}
\DeclareMathOperator{\mse}{mse}
\DeclareMathOperator{\se}{se}
\DeclareMathOperator{\range}{range}
\DeclareMathOperator{\dimension}{dim}
\DeclareMathOperator{\epi}{epi}
\DeclareMathOperator{\vecop}{vec}

\DeclareMathOperator{\real}{Re}
\DeclareMathOperator{\imag}{Im}

\DeclareMathOperator{\csum}{cs} % column sum
\DeclareMathOperator{\rsum}{rs} % row sum



\hypersetup{
    linkcolor=Blue,
    colorlinks=true,
    filecolor=magenta,      % color of file links
    urlcolor=cyan           % color of external links
}


%\pgfdeclareimage[height=1.2cm]{university-logo}{../tuxswatter2}
%\logo{\pgfuseimage{university-logo}}

%\addtobeamertemplate{headline}{}
%{%
%\begin{flushright}
%\begin{tikzpicture}[remember picture,overlay]
%\node [left ]{\includegraphics[width=0.5cm]{../tuxswatter2.png}};
%\end{tikzpicture}
%\end{flushright}
%\vskip -0.1cm
%} 

 \date[\today]{}

\title{Quantitative Economics Workshop Paris}




